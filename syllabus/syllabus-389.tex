\documentclass[12pt, a4paper]{article}

\usepackage{fontspec}
\usepackage{geometry}
\usepackage{lastpage}
\usepackage{fancyhdr}
\usepackage[hidelinks]{hyperref}
\usepackage[normalem]{ulem}
\usepackage{soul}
\usepackage{multicol}
\usepackage[dvipsnames]{xcolor}

\geometry{
  top=1.5cm,
  bottom=1.5cm,
  left=2cm,
  right=2cm,
  marginparsep=4pt,
  marginparwidth=1cm
}

\definecolor{lgrey}{rgb}{0.8, 0.8, 0.8}

\renewcommand{\headrulewidth}{0pt}
\pagestyle{fancyplain}
\fancyhf{}
% \lfoot{\color{lgrey}2020-12-01}
% \rfoot{\color{lgrey}page \thepage\ of \pageref{LastPage}}

\setlength{\parindent}{0pt}
\setlength{\parskip}{0pt}

\usepackage{xunicode}
\defaultfontfeatures{Mapping=tex-text}

\setromanfont{YaleNew}

\begin{document}

\begin{center}
\textbf{MATH/STAT 389: Statistical Learning, Spring 2021, ONLINE}
\end{center}

\noindent
\begin{tabular}{ l l }
\textbf{Instructor:} &  \textbf{Taylor Arnold} \\
E-mail: & \texttt{tarnold2@richmond.edu} \\
Website: & \texttt{https://statsmaths.github.io/stat389-f21} \\
Office Hours: & Following Class or By Appointment
\end{tabular}

\vspace{0.5cm}

\textbf{Format:} \vspace{6pt}

The course will be taught remotely over Zoom. During the first week of the
semester students will be placed into teams of around 4 people each. A large
portion of the course work will be done using break-out groups within these
teams. Materials and assignments for the class will be posted on the course
website. All assignments will be submitted through shared Google Drive folders.

\vspace{12pt}

\textbf{Class Components:} \vspace{6pt}

Course meetings will typically start with a short introduction to a new topic.
We will then use breakout rooms to split into groups. Each group will then
collectively work through questions posed as a R markdown notebook.
Teams will work together by picking a `driver' (which should alternate between
classes) to share their screen and take notes or write/run code. At the end of
class, the notes or code should be uploaded to the team's shared Google Drive
folder.

\vspace{12pt}

\textbf{Projects:} \vspace{6pt}

The core output of the course will be the completion of four projects. These
will be assigned throughout the course of the semester and focus on the
application of the course material to various textual corpora. Projects will
typically be completed together in your class group. There will
usually be plenty of class time to work on each project. We will also have
time to present a subset of the projects to the rest of the class.

\vspace{12pt}

\textbf{Attendence:} \vspace{6pt}

Given the amount of group-work required during the course meetings, it is
important for students to make an attempt to attend every class. With the
exception of extended absenses due to serious illness or
family emergencies, students will recieve a reduction in their final grade
(including a failing grade) after missing more than two class meetings.

\vspace{12pt}

\textbf{Final Grades:} \vspace{6pt}

Final grades will be computed by weighting the following course elements:

\begin{itemize}\setlength\itemsep{0em}
\item Classwork (notes and code submitted by the group), 20\%
\item Projects, 80\% (weighted evenly)
\end{itemize}

No specific grades for the classwork will be given. Students or groups that are
failing to achieve full marks will receive a warning during the semester.

\vspace{12pt}

\textbf{Getting Help:} \vspace{6pt}

Given social distancing rules, I find the easiest way to answer questions
is by finding time before, during, or after our scheduled class meeting times.
I will always make a habit of staying after class as long as students have
questions. If you want to meet right before class, just let me know.
If you have extended questions or personal concerns, please feel free
to email me or ask in class for a dedicated appointment.

\end{document}
